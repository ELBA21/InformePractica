\documentclass[12pt]{report}
\def\TITULO{Seleccion y Eticado para modelo Inteligencia Artificial}
\def\subtitulo{Práctica Intermedia}
\def\autora{Benjamin Alberto Martines Hernandez}
\def\correoa{benjaminalberto.martinez@alumnos.ulagos.cl}
\def\asignatura{Práctica Intermedia}
\def\campus{Campus Puerto Montt}
\def\carrera{Ingeniería Civil en Informática}
\input{Macros}
%LENGUAJE DE PROGRAMACIÓN POR DEFECTO
\lstset{language=Python}


\begin{document}
	
	%%%%%%%%%%%PORTADA%%%%%%%%%%%%%%%%%%%%%
%\setlength{\unitlength}{1 cm} %Especificar unidad de trabajo
\maketitle

\cleardoublepage
\pagenumbering{roman}
\setcounter{page}{1}

%INDICE GENERAL
\tableofcontents
%INDICE DE FIGURAS
\listoffigures
%INDICE DE TABLAS
\renewcommand{\listtablename}{Índice de tablas}\listoftables
%%%%%%%%%%%%%FIN PORTADA%%%%%%%%%%%%%%%%
\renewcommand{\lstlistlistingname}{Índice de algoritmos}
\lstlistoflistings
%\addcontentsline

%\thispagestyle{empty}
%\begin{abstract}
%\end{abstract}
\cleardoublepage
\pagenumbering{arabic}
\setcounter{page}{1}

\chapter{Fundamentación}
\section{Introducción}
Lorem ipsum dolor sit amet, consectetur adipiscing elit, sed do eiusmod tempor incididunt ut labore et dolore magna aliqua \cite{001}. Ut enim ad minim veniam, quis nostrud exercitation ullamco laboris nisi ut aliquip ex ea commodo consequat. Duis aute irure dolor in reprehenderit in voluptate velit esse cillum dolore eu fugiat nulla pariatur. Excepteur sint occaecat cupidatat non proident, sunt in culpa qui officia deserunt mollit anim id est laborum \cite{001,002}.
\subsection{Objetivos de la práctica}

Es importante  diferenciar entre el proceso formativo de las prácticas profesionales (incluidos en el D.U. de prácticas y especificados a continuación) de los objetivos del trabajo de práctica en sí.

Luego de los objetivos a continuación se dan ejemplos de objetivos generales y específicos para un trabajo de práctica en particular.


\subsubsection{General}

BORRAR EL OBJETIVO DE LA PRACTICA QUE NO CORRESPONDE.

\textbf{Práctica Intermedia\\} Permitir que la o el estudiante se familiarice con el ámbito laboral, desarrollando las actividades en contacto constante con el centro de práctica.

\textbf{Práctica Profesional\\} Permitir que la o el estudiante desempeñe labores relevantes que una o un Ingeniera/o Civil en Informática realizará en una organización

\subsubsection{Específicos\\}

\textbf{Práctica Intermedia}
\begin{enumerate}\justifying
  \item     Conocer la estructura de la organización y formas de operación de las áreas que componen el Centro de Práctica.
    \item  Conocer y aprender métodos de trabajo y formas de comunicación utilizados en entornos multidisciplinarios y/o multiculturales. 
    \item  Visualizar y proponer el método de ingeniería para concebir, diseñar, implementar y operar actividades colaborativas en la solución de problemas, en áreas informáticas o afines a esta, en el Centro de Práctica en que se desenvuelva.
\end{enumerate}

\textbf{Práctica Profesional}
\begin{enumerate}\justifying
  \item     Conocer la estructura de la organización y formas de operación de cada una de las áreas que componen el Centro de Práctica en las que la o el estudiante desempeñe sus tareas.
    \item Insertarse en grupos multidisciplinarios y/o multiculturales, participando activamente en  equipos de trabajo, con capacidad de adaptación, haciendo uso de habilidades interpersonales de comunicación, proponiendo ideas de solución a problemas y utilizando un pensamiento crítico en la toma de decisiones.
    \item Aprender métodos de trabajo, tendientes a la operación, administración y gestión, en áreas informáticas o similares.
    \item Concebir, diseñar, implementar y/o operar soluciones a problemas propuestos o detectados en el centro de práctica en que se desenvuelva, desde el enfoque de la Ingeniería Civil en Informática.
\end{enumerate}





\section{Objetivos del Trabajo de Práctica}
\subsubsection{General}
Debe tener relación con los objetivos de la práctica. Estos serán los objetivos propios de su práctica.
%Mejorar el proceso de seguimiento del desempeño académico de estudiantes de Educación Media en las distintas instituciones de Educación pública de la comuna de Osorno.
\subsubsection{Específicos\\}
\begin{enumerate}\justifying
  \item Lorem ipsum dolor sit amet, consectetur adipiscing elit, sed do eiusmod tempor incididunt ut labore et dolore magna aliqua.
  \item Lorem ipsum dolor sit amet, consectetur adipiscing elit, sed do eiusmod tempor incididunt ut labore et dolore magna aliqua.
\end{enumerate}











\section{Datos del Alumno}
\begin{description}\justifying
  \item [Nombre del alumno] Lorem ipsum
  \item [Año de ingreso a la Universidad] 20XX
  \item [Campus] 
  \item [Tipo   de práctica] Practica X
  \item [Fecha de realización de la práctica] XX de XXXXX de 20XX - XX de XXXXX de 20XX
\end{description}


\chapter{Empresa}
\section{Datos de la Empresa}

\begin{description}\justifying
  \item [Nombre de la empresa] Lorem ipsum
  \item [RUT] Lorem ipsum
  \item [Representante Legal] Lorem ipsum
  \item [Rubro] Lorem ipsum
  \item [Dirección] Lorem ipsum
  \item [Sitio Web] Lorem ipsum
  \item [Teléfono] +56XX-XXXX-XXXX
  \item [Nombre del supervisor] Lorem ipsum
  \begin{description}\justifying
  \item[Cargo] Lorem ipsum
\end{description}
  \item [Sección en la que desarrolló la práctica] Lorem ipsum
  \item [Relación familiar con alguien de la empresa] (SI/NO)
\end{description}


\section{Descripción de la Empresa}
Incluye una descripción general de la empresa, como organización, rubro, localización y también descripción particular del departamento en que realizó la práctica. 

Se puede incluir un cuadro comparativo o descriptivo como la Tabla \ref{tab1}.

\begin{table}[htb]
\centering
\begin{tabular}{|c|c|c|}\hline
  & & \\
  \hline\hline
  & & \\
  & & \\\hline
\end{tabular}
\caption{Ejemplo Tabla}
\label{tab1}
\end{table}

\subsection{Organigrama}
Describir el organigrama que se ve en la Figura \ref{organigrama}.

\begin{figure}[H]
\centering
 \includegraphics[scale=0.1]{portada}
  \caption{Organigrama de la empresa \dots}
  \label{organigrama}
\end{figure}

\section[Descripción del Área]{Descripción del área donde desarrolló la práctica}
Describir el área que se ve en la Figura \ref{organigramaarea}.

\begin{figure}[H]
\centering
 \includegraphics[scale=0.1]{portada}
  \caption{Organigrama del área \dots}
  \label{organigramaarea}
\end{figure}

\section{Resumen de actividades de la Práctica}\label{actividades}

Debe tener relación con los objetivos (Ambos).

\chapter{Desarrollo de la Práctica}

\section{Descripción del Proyecto o Actividad} 

\rojo{Solo si corresponde, obligatorio en la práctica verano II}. Describa el proyecto en que participó, indicando objetivos y resultados esperados. Relacionado al punto c) del articulo 4to.

\begin{quote}
    Detectar, proponer y desarrollar soluciones a problemas desde una visión de Ingeniería Civil en Informática.
\end{quote}

\section{Descripción de Actividades Realizadas}
Debe indicar detalladamente que actividades se realizaron, además de tareas específicas, áreas involucradas de la empresa, herramientas y plataformas usadas, resultados obtenidos.

Añadir imágenes y códigos.

\chapter{Discusión y Opinión Personal}
Lorem ipsum dolor sit amet, consectetur adipiscing elit, sed do eiusmod tempor incididunt ut labore et dolore magna aliqua.


\chapter{Conclusión}

Debe indicar los beneficios obtenidos por la empresa producto de su cometido, así como los beneficios personales obtenidos por usted en cuanto a lo aprendido, sean estos conocimientos, habilidades o aptitudes (considerando el perfil de la carrera).

%%%%%
\renewcommand{\refname}{Referencias}

%agregar referencias
\bibliographystyle{IEEEtran}
%\nocite{*} % mostrar todas las referencias aunque no esten citadas
\bibliography{bibliografia}

%\input{Ejemplos}%comentar/eliminar esta línea
\end{document}
