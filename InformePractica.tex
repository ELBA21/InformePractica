\documentclass[12pt]{report}
\def\TITULO{Seleccion y Eticado para modelo Inteligencia Artificial}
\def\subtitulo{Práctica Intermedia}
\def\autora{Benjamin Alberto Martines Hernandez}
\def\correoa{benjaminalberto.martinez@alumnos.ulagos.cl}
\def\asignatura{Práctica Intermedia}
\def\campus{Campus Puerto Montt}
\def\carrera{Ingeniería Civil en Informática}
\input{Macros}
%LENGUAJE DE PROGRAMACIÓN POR DEFECTO
\lstset{language=Python}


\begin{document}
	
	%%%%%%%%%%%PORTADA%%%%%%%%%%%%%%%%%%%%%
%\setlength{\unitlength}{1 cm} %Especificar unidad de trabajo
\maketitle

\cleardoublepage
\pagenumbering{roman}
\setcounter{page}{1}

%INDICE GENERAL
\tableofcontents
%INDICE DE FIGURAS
\listoffigures
%INDICE DE TABLAS
\renewcommand{\listtablename}{Índice de tablas}\listoftables
%%%%%%%%%%%%%FIN PORTADA%%%%%%%%%%%%%%%%
\renewcommand{\lstlistlistingname}{Índice de algoritmos}
\lstlistoflistings
%\addcontentsline

%\thispagestyle{empty}
%\begin{abstract}
%\end{abstract}
\cleardoublepage
\pagenumbering{arabic}
\setcounter{page}{1}

\chapter{Fundamentación}
\section{Introducción}
Este informe pretende describir el conjunto de conocimientos adquiridos por medio de la realización de mi practica intermedia realizada en la Universidad de Los Lagos  

COMPLETAR MAS !!!!!!!!!!!!!!!!!!!!!!!!!!!!!!!!!!!!!!!!!!!!!
\subsection{Objetivos de la práctica}

Es importante  diferenciar entre el proceso formativo de las prácticas profesionales (incluidos en el D.U. de prácticas y especificados a continuación) de los objetivos del trabajo de práctica en sí.

Luego de los objetivos a continuación se dan ejemplos de objetivos generales y específicos para un trabajo de práctica en particular.


\subsubsection{General}

\textbf{Práctica Intermedia\\} Permitir que la o el estudiante se familiarice con el ámbito laboral, desarrollando las actividades en contacto constante con el centro de práctica.

\subsubsection{Específicos\\}

\textbf{Práctica Intermedia}
\begin{enumerate}\justifying
  \item     Conocer la estructura de la organización y formas de operación de las áreas que componen el Centro de Práctica.
    \item  Conocer y aprender métodos de trabajo y formas de comunicación utilizados en entornos multidisciplinarios y/o multiculturales. 
    \item  Visualizar y proponer el método de ingeniería para concebir, diseñar, implementar y operar actividades colaborativas en la solución de problemas, en áreas informáticas o afines a esta, en el Centro de Práctica en que se desenvuelva.
\end{enumerate}


\section{Objetivos del Trabajo de Práctica}
\subsubsection{General}
Apoyar el desarrollo del proyecto de investigación mediante la preparación y optimización de un conjunto de datos para el entrenamiento de una inteligencia artificial, contribuyendo a la validación manual de etiquetas y al desarrollo de herramientas que faciliten su corrección.
\subsubsection{Específicos\\}
\begin{enumerate}\justifying
  \item Detección y marcado manual de inconsistencias en las mascaras generadas automáticamente sobre imágenes de salmón "coho", con el fin de asegurar su correspondencia con la silueta del pez vista en la fotografía.
  \item Generación de un programa que transforme las mascaras aun formato legible por la herramienta  "Label Studio" facilitando su revisión y corrección.
  \item Colaborar en la preparación del conjunto de datos y comprender el flujo de trabajo asociado al entrenamiento de modelos de inteligencia artificial en contextos aplicados.
\end{enumerate}











\section{Datos del Alumno}
\begin{description}\justifying
  \item [Nombre del alumno] Benjamin Albertio Martinez Hernandez
  \item [Año de ingreso a la Universidad] 2023
  \item [Campus] Puerto Montt
  \item [Tipo   de práctica] Practica Intermedia
  \item [Fecha de realización de la práctica] 09 de MAYO de 2025 - 31 de AGOSTO de 2025
\end{description}


\chapter{Empresa}
\section{Datos de la Empresa}

\begin{description}\justifying
  \item [Nombre de la empresa] Universidad de Los Lagos
  \item [RUT] 70772100-6
  \item [Representante Legal] Lorem ipsum
  \item [Rubro] Educación - Investigación
  \item [Dirección] Avda Fuchslocher 1305 Osorno
  \item [Sitio Web] https://www.ulagos.cl/
  \item [Teléfono] +56
  \item [Nombre del supervisor] Joel Torres
  \begin{description}\justifying
  \item[Cargo] Investigador
\end{description}
  \item [Sección en la que desarrolló la práctica] Investigacion y Desarrollo
  \item [Relación familiar con alguien de la empresa] NO
\end{description}


\section{Descripción de la Empresa}
La Universidad de Los Lagos es una institución estatal de educación superior que promueve la formación académica integral, la investigación científica y la vinculación con el medio. Su labor se centra especialmente en atender las necesidades y problemáticas del territorio sur austral de Chile, contribuyendo al desarrollo local y regional.

%\begin{table}[htb]
%\centering
%\begin{tabular}{|c|c|c|}\hline
%  & & \\
%  \hline\hline
%  & & \\
%  & & \\\hline
%\end{tabular}
%\caption{Ejemplo Tabla}
%\label{tab1}
%\end{table}

\subsection{Organigrama}
Describir el organigrama que se ve en la Figura \ref{organigrama}.

\begin{figure}[H]
\centering
 \includegraphics[scale=0.1]{portada}
  \caption{Organigrama de la empresa \dots}
  \label{organigrama}
\end{figure}

\section[Descripción del Área]{Descripción del área donde desarrolló la práctica}
La práctica profesional se llevó a cabo en el área de Investigación y Desarrollo de la Universidad de Los Lagos, específicamente en un proyecto del Departamento de Acuicultura y Recursos Agroalimentarios.

Esta unidad tiene como objetivo promover la investigación aplicada y el desarrollo de nuevas tecnologías vinculadas al rubro de la salmonicultura, integrando herramientas informáticas y modelos de inteligencia artificial para optimizar procesos de análisis y clasificación de imágenes biológicas.

Dentro de este contexto, mi labor se orientó principalmente al procesamiento y revisión de imágenes de salmones, tarea esencial para la validación de modelos de segmentación automática. Este trabajo permitió generar conjuntos de datos de alta calidad para etapas posteriores de entrenamiento y evaluación de modelos de visión por computadora.

%\begin{figure}[H]
%\centering
% \includegraphics[scale=0.1]{portada}
%  \caption{Organigrama del área \dots}
%  \label{organigramaarea}
%\end{figure}

\section{Resumen de actividades de la Práctica}\label{actividades}

Durante el desarrollo de mi práctica profesional participé principalmente en actividades de procesamiento de datos e inspección visual de imágenes. Entre las tareas más relevantes se incluyen:
\begin{itemize}
	\item Revisión y corrección manual de más de 90.000 imágenes de salmones segmentadas mediante modelos de inteligencia artificial.
	\item Identificación de errores en las máscaras generadas automáticamente y preparación de los conjuntos de datos corregidos.
	\item Colaboración en jornadas de captura de nuevas imágenes junto al equipo de investigación, apoyando el proceso de toma de muestras fotográficas para ampliar el conjunto de datos.
	\item Apoyo en la planificación de futuras integraciones con herramientas como Label Studio para la mejora del flujo de anotación.
\end{itemize}

\chapter{Desarrollo de la Práctica}

Durante mi práctica intermedia, participé en un proyecto de investigación desarrollado por el Departamento de Acuicultura y Recursos Agroalimentarios de la Universidad de Los Lagos, enfocado en la aplicación de inteligencia artificial y visión computacional para el análisis de imágenes de salmones.
El objetivo principal del proyecto fue mejorar la calidad de los datos utilizados para entrenar modelos de segmentación automática, revisando y corrigiendo errores presentes en las máscaras generadas por la IA.
El trabajo se llevó a cabo bajo el área de Investigación y Desarrollo de la universidad, en colaboración con un equipo técnico y académico encargado de la validación de los modelos.



\section{Descripción del Proyecto o Actividad} 

Durante el desarrollo de la práctica, participé en diversas tareas relacionadas con el procesamiento y validación de imágenes. Entre las principales actividades se incluyen:

\begin{itemize}
	\item Revisión y corrección manual de más de 90.000 imágenes de salmones segmentadas por modelos de inteligencia artificial.
	\item Identificación de errores en las máscaras automáticas y apoyo en la generación de nuevos conjuntos de datos corregidos.
	\item Colaboración en salidas a terreno junto al equipo de investigación para la captura de nuevas imágenes, ampliando el conjunto de datos disponible.
	\item Apoyo en la exploración de herramientas de anotación como Label Studio, destinadas a optimizar el flujo de corrección de datos en futuras etapas del proyecto.
\end{itemize}

Estas actividades permitieron fortalecer habilidades en organización de datos, control de calidad de información visual y trabajo en equipo dentro de un entorno de investigación aplicada, contribuyendo al desarrollo de un proyecto con impacto en el ámbito acuícola.

\section{Descripción de Actividades Realizadas}
\subsection{Entorno de trabajo y herramientas}
El desarrollo se realizó en un entorno controlado dispuesto por la universidad, con acceso remoto a través de la herramienta AnyDesk. Debido al carácter confidencial del proyecto y las políticas de protección de datos, todo el trabajo se efectuó directamente sobre un equipo designado, sin uso de servicios en la nube.

El entorno principal se basó en Python, utilizando librerías orientadas al manejo y visualización de imágenes científicas en formato .npy. Para la gestión de datos, se empleó una base local en MongoDB, mientras que la manipulación y revisión de las máscaras asociadas se realizó mediante una herramienta de visualización interna desarrollada previamente en Python.
\subsection{Metodo de trabajo y comunicación}

 \subsection{Actividades principales}
 \begin{itemize}
 	\item Revisión manual de más de 90.000 imágenes segmentadas mediante modelos de inteligencia artificial, verificando la correspondencia entre la máscara y el individuo representado.
 	\item Corrección de errores comunes en las máscaras, como desplazamientos, recortes parciales o segmentaciones múltiples no deseadas.
 	\item Clasificación y reorganización de los conjuntos de datos para su posterior uso en etapas de entrenamiento y validación de modelos.
 	\item Apoyo en la toma de nuevas imágenes junto al equipo de investigación, ampliando la base de datos disponible para futuras iteraciones del modelo.
 	\item Exploración de herramientas complementarias como Label Studio, orientadas a automatizar la validación y anotación manual de imágenes.
 \end{itemize}

\subsection{Aprendizaje y resultados}

Si bien el trabajo no involucró directamente programación avanzada, permitió adquirir experiencia práctica en procesamiento de datos, control de calidad de datasets y flujo de trabajo colaborativo en proyectos de visión computacional.
Además, fortaleció la comprensión del ciclo de vida de un modelo de IA, desde la captura y limpieza de datos hasta su preparación para el entrenamiento, junto con habilidades de organización, comunicación y precisión en tareas repetitivas de gran volumen.


\chapter{Discusión y Opinión Personal}
Lorem ipsum dolor sit amet, consectetur adipiscing elit, sed do eiusmod tempor incididunt ut labore et dolore magna aliqua.


\chapter{Conclusión}

Debe indicar los beneficios obtenidos por la empresa producto de su cometido, así como los beneficios personales obtenidos por usted en cuanto a lo aprendido, sean estos conocimientos, habilidades o aptitudes (considerando el perfil de la carrera).

%%%%%
\renewcommand{\refname}{Referencias}

%agregar referencias
\bibliographystyle{IEEEtran}
%\nocite{*} % mostrar todas las referencias aunque no esten citadas
\bibliography{bibliografia}

%\input{Ejemplos}%comentar/eliminar esta línea
\end{document}
